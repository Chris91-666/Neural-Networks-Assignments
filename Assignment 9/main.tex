\documentclass[a4paper]{article}
\usepackage[utf8]{inputenc}
\usepackage{fancyhdr}
\usepackage{vmargin}
\usepackage{listings}

%nicer tables
\usepackage{booktabs}

\usepackage{graphicx}

\usepackage{float}

\usepackage{color}
\usepackage{url}
\usepackage{hyperref}

\usepackage{enumerate}

\usepackage[backend=biber]{biblatex}

\usepackage{csquotes}

\usepackage{multicol}
\setlength{\columnsep}{1cm}
\setlength{\headheight}{36pt}

\definecolor{bluekeywords}{rgb}{0.13,0.13,1}
\definecolor{greencomments}{rgb}{0,0.5,0}
\definecolor{redstrings}{rgb}{0.9,0,0}



%
\usepackage{amsmath, amsthm, amssymb}
\usepackage[ngerman, english]{babel}
\usepackage{marvosym}
\usepackage{graphics}
\usepackage{extarrows}
\usepackage{forloop}
\usepackage{mathtools}

\usepackage[]{algorithm2e}

\usepackage{hyperref}% http://ctan.org/pkg/hyperref
\usepackage{cleveref}% http://ctan.org/pkg/cleveref
\usepackage{lipsum}% http://ctan.org/pkg/lipsum
\newtheorem{definition}{Definition}
\newtheorem{theorem}{Theorem}
\newtheorem{lemma}{Lemma}
\newtheorem{preliminary}{Preliminary}
\newtheorem{notation}{Notation}
\newtheorem{property}{Property}
\newtheorem{corollary}{Corollary}
\newtheorem{example}{Example}
\newtheorem{hypothesis}{Hypothesis}

\crefname{theorem}{Theorem}{Theorems}
\crefname{definition}{Definition}{Definitions}
\crefname{lemma}{Lemma}{Lemmas}
\crefname{preliminary}{Preliminary}{Preliminaries}
\crefname{notation}{Notation}{Notations}
\crefname{property}{Property}{Properties}
\crefname{corollary}{Corollary}{Corollaries}
\crefname{example}{Example}{Examples}
\crefname{hypothesis}{Hypothesis}{Hypotheses}

\newenvironment{beweis}{\begin{proof}[Beweis]}{\end{proof}}
%



\lstset{language=Python,
showspaces=false,
showtabs=false,
breaklines=true,
showstringspaces=false,
breakatwhitespace=true,
escapeinside={(*@}{@*)},
commentstyle=\color{greencomments},
keywordstyle=\color{bluekeywords}\bfseries,
stringstyle=\color{redstrings},
basicstyle=\ttfamily
}


\setlength{\parindent}{0pt}
\setlength{\parskip}{5pt}

\frenchspacing
\pagestyle{fancy}
\sloppy 

\markright{headline}

\addbibresource{references.bib}

\begin{document}

\lhead{\begin{tabular}{l}
\\
Neural Networks\\
WiSe 2020/2021\\
\end{tabular}}

\rhead{\begin{tabular}{r}
Assignment 9\\
Simon Laurent Lebailly, 2549365, s9sileba@teams.uni-saarland.de\\%% <=== Also HERE if you have a team mateUpdate Name HERE !!! 
Christian Mathieu Schmidt, 2537621, s9cmscmi@teams.uni-saarland.de
\end{tabular}}




\section*{Exercise 9.1: Convolutions}
    \subsection*{a)}
        Let $f(x) = e^{-x}$ and $g(x) = sin(x)$.
        Then we can compute the convolution $f(x) \ast g(x)$ as follows
        \begin{align}
            f(x) \ast g(x) &= (f \ast g)(x) = (g \ast f)(x)\\
            &= \int\limits_{t=0}^x f(x-t) g(t) dt\\
            &= \int\limits_{t=0}^x e^{-x+t} sin(t) dt\\
            &= \int\limits_{t=0}^x e^{-x} e^{t} sin(t) dt\\
            &= e^{-x} \cdot \int\limits_{t=0}^x e^{t} sin(t) dt
        \end{align}
        To solve the integral we use \textbf{two times partial integration:} $\int m'n = mn - \int mn'$\\
        \textbf{First time:} We set $m(t) = e^t$, $n(t) = sin(t)$, and become $m'(t) = e^t$, $n'(t) = cos(t)$
        \begin{align}
            \int\limits_{t=0}^x e^{t} sin(t) dt &= \left[ e^{t} sin(t) \right]_{t=0}^x - \int\limits_{t=0}^x e^{t} cos(t) dt
        \end{align}
        \textbf{Second time:} We set $m(t) = e^t$, $n(t) = cos(t)$, and become $m'(t) = e^t$, $n'(t) = -sin(t)$
        \begin{align}
            \int\limits_{t=0}^x e^{t} cos(t) dt &= \left[ e^{t} cos(t) \right]_{t=0}^x - \int\limits_{t=0}^x - e^{t} sin(t) dt\\
            &= \left[ e^{t} cos(t) \right]_{t=0}^x + \int\limits_{t=0}^x e^{t} sin(t) dt
        \end{align}
        Now we put the result of the two partial integrations together, and get the following equation
        \begin{align}
            & & \int\limits_{t=0}^x e^{t} sin(t) dt &= \left[ e^{t} sin(t) \right]_{t=0}^x - \left( \left[ e^{t} cos(t) \right]_{t=0}^x + \int\limits_{t=0}^x e^{t} sin(t) dt \right)\\
            & & &= \left[ e^{t} \left( sin(t) - cos(t) \right) \right]_{t=0}^x - \int\limits_{t=0}^x e^{t} sin(t) dt\\
            &\Leftrightarrow & 2 \cdot \int\limits_{t=0}^x e^{t} sin(t) dt &= \left[ e^{t} \left( sin(t) - cos(t) \right) \right]_{t=0}^x\\
            &\Leftrightarrow & \int\limits_{t=0}^x e^{t} sin(t) dt &= \frac{1}{2} \left[ e^{t} \left( sin(t) - cos(t) \right) \right]_{t=0}^x
            = \frac{1}{2} \left( e^{x} \left( sin(x) - cos(x) \right) + 1 \right)
        \end{align}
        If we now put all together, we yield
        \begin{align}
            f(x) \ast g(x) &= (f \ast g)(x) = (g \ast f)(x)\\
            &= e^{-x} \cdot \frac{1}{2} \left( e^{x} \left( sin(x) - cos(x) \right) + 1 \right)
            = \frac{1}{2} \left( sin(x) - cos(x) + e^{-x} \right)
        \end{align}
        

    \subsection*{b)}
        Let $f(x) = \frac{1}{\sqrt{2 \pi}} e^{-\frac{x^2}{2}}$ and $F^2(x) = (f \ast f)(x)$.
        \textbf{To show:} $F^{\infty}(x) = 0$
        \begin{proof}
            We know that $f(x) = \frac{1}{\sqrt{2 \pi}} e^{-\frac{x^2}{2}}$ is the density function of the gaussian distribution 
            $\mathcal{N}(\mu,\,\sigma^{2})$ with mean $\mu = 0$ and standard deviation $\sigma^2 =1$, 
            also called density function of the "standard normal distribution" $\mathcal{N}(0,1)$.\\
            For the density function of the standard normal distribution it holds
            \begin{align}
                \int\limits_{-\infty}^{\infty} f(x) dx 
                &= \int\limits_{-\infty}^{\infty} \frac{1}{\sqrt{2 \pi}} e^{-\frac{x^2}{2}} dx = 1
            \end{align}
            So the convolution $F^{\infty}(x)$ can be computed as follows:
            \begin{align}
                F^{\infty}(x) &= \lim_{n \rightarrow \infty} F^n(x)\\
                &= \lim_{n \rightarrow \infty} \left( \frac{1}{\sqrt{2 \pi}} \right)^n \cdot G^n(x)
            \end{align}
            with $G^n(x) = (\underbrace{g \ast ... \ast g}_{n\text{ times}})(x)$ and $g(x) = e^{-\frac{x^2}{2}}$, we can assume
            $0 < g(x) < 1$ for all $x /in \mathbb{R}\backslash\{0\}$,\\ and equal one for $x=0$.\\
            We can follow that $0 < G^{n+1}(x) < G^{n} < \int\limits_{-\infty}^{\infty} g(x) dx = \sqrt{2\pi}$ for any $n \in \mathbb{N}$, and thus
            \begin{align}
                G^{\infty}(x) &= \lim_{n \rightarrow \infty} G^n(x) = 0
            \end{align}
            And so we can conclude
            \begin{align}
                F^{\infty}(x) &= \lim_{n \rightarrow \infty} \left( \frac{1}{\sqrt{2 \pi}} \right)^n \cdot G^n(x)\\
                &= \underbrace{\lim_{n \rightarrow \infty} \left( \frac{1}{\sqrt{2 \pi}} \right)^n}_{=0} \cdot \underbrace{\lim_{n \rightarrow \infty} G^n(x)}_{=0\ (18)}
                = 0 \cdot 0 = 0
            \end{align}
            
        \end{proof}

\newpage
    \subsection*{c)}
        \begin{itemize}
            \item \textbf{Dimension output feature map:}\\
                \begin{align}
                    n_{out} &= \left\lfloor \frac{n_{in} + 2*p - k}{s} \right\rfloor + 1
                    = \left\lfloor \frac{100 + 2*0 - 7}{2} \right\rfloor + 1\\
                    &= \left\lfloor \frac{93}{2} \right\rfloor + 1
                    = \left\lfloor 46,5 \right\rfloor + 1 = 46 + 1 = 47
                \end{align}
                with $n_{in}$ as number of input features, $n_{out}$ as number of output features, $k$ as convolution kernel size, 
                $p$ as convolution padding size, and $s$ as convolution stride size.\\
                Therefore, the output feature map has the following dimension:
                $$47 \times 47 \times 64$$
            
            \item \textbf{Total number of weight parameters $n_w$:}\\
                \begin{align}
                    n_w &= \underbrace{(7 * 7)}_{\text{Weights from the kernel}} *\ \ \ \ \ \ \ \ \ \  (\underbrace{32}_{\text{Input channels}} * \underbrace{2}_{\text{Filters}})\\ 
                    &= 49\ \ \ \ \ \ \ \  * \underbrace{64}_{\text{Output channels}}\\ 
                    &= 3136
                \end{align}

        \end{itemize}

    \subsection*{d)}
        \subsubsection*{i)}
            With correlation we can test for similarity directly between filter and image section in the order in which the pixels are. With convolution the filter is practically flipped.\\
            For symmetrical filters there is no difference in the output, for all others it is like flipping the input image at the second main diagonal first and then performing convolution.

        \subsubsection*{ii)}
            It changes the order in which the weights are learned, but not the way they are learned.
            In the end, it does not change anything essential when learning the weights.



\section*{Exercise 9.2: Batch Normalization}
    


\section*{Exercise 9.3: CNN}
    See Jupyter Notebook file!



\end{document}
